\documentclass{article}
\usepackage[margin=1in]{geometry}
\usepackage{url}
\usepackage{color}
\usepackage[dvipsnames]{xcolor}
\usepackage{listings}
\usepackage{palatino}

\newcommand{\red}[1]{{\color{red}{\emph{#1}}}}
\newcommand{\projectName}{\texttt{QBASHER}}

\title{Introduction to QBASHER\\
  \large ~\\
  Version 1.5.123-OS}
\author{Microsoft}

\begin{document}
\maketitle{}


QBASHER is a stand-alone indexing and querying system, based on an
inverted file format optimized for short texts.  Example short text
collections include academic paper titles, song lyric lines, and logs
of search queries.  QBASHER is capable of supporting
word-order-independent query auto-suggest, query
classification, and simple retrieval.

QBASHER is designed to handle many billions of records in a single
index and to achieve high query processing rates and low latency query
responses.  Data to be indexed is expected to be in TSV
(tab-separated-value) format with a
minimum of two columns.  Column one contains the document text, column
two is expected to contain a numeric quantity representing the static
score (popularity, authority, etc) of the document. Extra columns
may contain additional application-specific information such as
latitude and longitude, object-id within a knowledge base, or a
display-oriented form of the document.   In normal operation, QBASHER
internally reorders documents into order of descending static score,
after normalising the scores into the range $0-1$.  During retrieval,
candidates are ranked by a linear combination $\alpha - \theta$ of
features, including the static score.  One of the features is a BM25
score. In a query classification scenario, different features are
used: $\omega - \chi$.

In order to calculate features used in scoring candidates and to
perform matching of word prefixes and rank-only words, QBASHER uses
the raw document text as its own ``per-document-index''.  In other
words, features are checked or calculated using string scanning.  This
makes sense given the assumed shortness of the documents.  To cut down
on the amount of scanning required a Bloom filter based on word first
letters is used.

QBASHER has three essential components:

\begin{description}
\item [QBASHI] Indexer executable
  \item [QBASHQ-LIB] Library implementing the query processing API for
    both C and C\# p-invoke front-ends.
    \item [QBASHQ or QBASHQsharpNative] Query processing front-ends
      in C and C\# respectively which process queries using the
      QBASHQ-LIB libraries.
\end{description}

    

QBASHER supports the following query processing operators:

\vspace{3mm}
\begin{tabular}{|r|l|}
  \hline
\texttt{"..."}& Phrase match, e.g. \texttt{"san francisco"} \\
\texttt{[...]}& Disjunction, e.g. \texttt{[cat dog] food}\\
\texttt{\~}& Rank-only operator, e.g. \verb|queens of england ~victoria|\\
\texttt{/}& Match a word prefix, e.g. \texttt{Seattle /seah}\\
\verb|<|& Match a line prefix, e.g. \texttt{<Faceb}\\
Phrase within disjunction& e.g. \texttt{["mountain lion" "snow
    leopard"]} \\
Disjunction within phrase& e.g. \texttt{"[metal plastic] lid"}\\
\hline
\end{tabular}
\vspace{3mm}

QBASHER is written in C and can be built with either GCC or Visual
Studio 2015.  The query processing logic is implemented in a DLL
which has an API, designed to be ``p-invoked'' from a C-sharp main program.
An example C-sharp main program is included for illustration and testing
purposes.

The \texttt{QBASHER/scripts} directory contains a reasonably
extensive suite of test scripts, written in \texttt{perl5}.  These
tests must pass prior to checking in any modifications to the code.


\section{Getting going with \projectName}

\subsection{Obtaining \projectName}

Assuming you are working with a Unix style shell, e.g. on Linux or MacOSx,
under Cygwin, or under Ubuntu Bash for Windows 10 ....

\begin{verb}
git clone "https://github.com/Microsoft/QBASHER"
\end{verb}


\subsection{Prerequisites}

To use the system you need first to build executables from their sources.
For this you need to have installed either \texttt{gcc}\footnote{The
  MacOS version of \texttt{gcc} which is actually \texttt{Clang}, see
  \url{https://clang.llvm.org/comparison.html} also works.}  and
\texttt{make} or Microsoft Visual Studio 2015.  

To use the system you will also need to have \texttt{perl 5}
installed.  Finally, to re-generate the PDF of this document you will
need a version of LaTeX.

\subsection{Building with gcc / make}

If you have \texttt{gcc}, \texttt{make}, and \texttt{perl5} installed
and in your path you can just type:

\begin{verbatim}
cd QBASHER/src    # Change to the src directory of the cloned repository
make cleanest
make
\end{verbatim}

Later, you can save some time by using 'make cleaner' instead of 'make
cleanest' -- it avoids the rebuilding of the PCRE2 libraries.

Libraries and executables will appear in the QBASHER/src directory.

After successfully building the libraries and executables, you should
run the test suite, following the instructions in
\texttt{QBASHER/scripts/README.txt}. See
\texttt{QBASHER/test\_queries/README.txt} for information about the
sets of test queries used by the test suite.


\subsection{Building with Visual Studio 2015}

To build the executable using Visual Studio 2015 (VS 2015), please navigate to
\texttt{QBASHER/src/visual\_studio} using the file explorer, then
double click on \texttt{visual\_studio.sln}. That should open the VS
2015 application.  In the lower part of the toolbar at the top
of the VS window, select \texttt{Release} and \texttt{x64} from the
first and second drop-downs.  (The QBASHER design assumes 64-bit.
You will need to change a number of configuration parameters if you
wish to build a Debug version.)
Then click on \texttt{Rebuild solution}
from the \texttt{Build} menu.  

\noindent Assuming the build succeeds you should see:

\begin{verbatim}
========== Rebuild All: 9 succeeded, 0 failed, 0 skipped ==========
\end{verbatim}

\noindent in the output pane, and all the EXEs and DLLs will be in:

\begin{verbatim}
QBASHER/src/visual_studio/x64/Release/
\end{verbatim}

You should then run the test suite, following the instructions in
\texttt{QBASHER/scripts/README.txt}. See
\texttt{QBASHER/test\_queries/README.txt} for information about the
sets of test queries used by the test suite.

\section{Help on options}
If you run any of the .EXE files from the command line, without
specifying any options, that EXE will print a list of all available
options, each with its default value and a one line description.
Options whose names start with ``x\_'' are considered experimental and
are more likely to be removed or changed in future releases.    At the
time of writing (12 Dec 2017) there are some such options which have
proved their worth and which have been thoroughly tested, but which
have not yet had the 'x\_' dropped.


\section{Examples of command-line usage}

In each of the following examples, it is assumed that the current
working directory is one level above the \texttt{src} directory, and
that there is a \texttt{data\_data/wikipedia\_titles} directory containing a file
\texttt{QBASH.forward} in TSV format with the records to be
indexed in column one, static weights in column two, text for display
in column three.


\subsection{Simplest indexing of a data set}

{\footnotesize
\begin{verbatim}
src/visual_studio/x64/Release/QBASHI.exe index_dir=data
\end{verbatim}
}

\noindent The \texttt{QBASH.forward} file is indexed, resulting in three
additional files being created in the same directory:
\texttt{QBASH.if}, \texttt{QBASH.vocab}, \texttt{QBASH.doctable}.
Remember that \texttt{QBASH.forward} is also part of the index and is
needed for query processing.  Note that, by default, the records in
\texttt{QBASH.forward} are internally (i.e. no change to the file
itself) sorted in descending static score order.

\subsection{Running a single query against the simplest index}
{\footnotesize
\begin{verbatim}
src/visual_studio/x64/Release/QBASHQ.exe
                index_dir=test_data/wikipedia_titles -pq="united states elections"
src/visual_studio/x64/Release/QBASHQ.exe
                index_dir=test_data/wikipedia_titles -pq="united states elections /pr"
src/visual_studio/x64/Release/QBASHQ.exe
               index_dir=test_data/wikipedia_titles -pq="united states elections pr" -auto_partials=ON
\end{verbatim}
}

\noindent In the first example, \texttt{QBASHQ} retrieves up to eight records which
contain all three words and ranks them in descending order of static
score:
{\footnotesize
\begin{verbatim}
Voter turnout in the United States presidential elections       0.62035
United States elections, 2016   0.60078
List of United States presidential elections by popular vote margin     0.56947
List of third party performances in United States elections     0.51663
List of United States presidential elections by Electoral College margin        0.48337
United States elections, 2014   0.48337
United States House of Representatives elections, 2012  0.47945
United States Senate elections, 2018    0.44031

\end{verbatim}
}

\noindent In the second example, the \texttt{/pr} indicates that ``pr'' is to be
treated as a word prefix rather than a word.   The set of records
containing ``united'', ``states'', and  ``elections'' are filtered to remove those which
don't contain a word beginning with ``pr''.  

{\footnotesize
\begin{verbatim}Voter turnout in the United States presidential elections       0.62035
List of United States presidential elections by popular vote margin     0.56947
List of United States presidential elections by Electoral College margin        0.48337
Electoral vote changes between United States presidential elections     0.37573
Lists of newspaper endorsements in United States presidential elections 0.33072
United States presidential elections in Missouri        0.24853
United States presidential elections    0.18591
Canada and the United States presidential elections     0.18591

\end{verbatim}
}

\noindent The third example produces exactly the same results as the second,
because the \texttt{auto\_partials} option treats the last word in the
query (unless followed by a space) as though it had a leading slash.


\subsection{Running a batch of queries against the simplest index}
{\footnotesize
\begin{verbatim}
src/visual_studio/x64/Release/QBASHQ.exe
                index_dir=test_data/wikipedia_titles -file_query_batch=test_queries/emulated_log_1k.q 
src/visual_studio/x64/Release/QBASHQ.exe
                index_dir=test_data/wikipedia_titles
                -file_query_batch=test_queries/emulated_log_1k.q -query_streams=1
\end{verbatim}
}

Assuming that \texttt{test\_queries/emulated\_log\_1k.q} contains a
set of queries, one per line, the first example will run all the
queries and print the results on standard output.   The queries will
be run using multiple (ten by default) threads, and the order of
results will not in general correspond to the order of queries in the
query batch.   In the second example, the \texttt{query\_streams}
option runs the queries single-threaded and presents results in the
same order as the queries appear in the file.

\section{Committing changes to \projectName}

Before committing changes, please make sure that at least the BASIC
tests in the test suite pass.  Then you can use \texttt{make git} to
issue all the necessary 'git add' commands to ensure that all the
necessary items (and none of the derivative items) are added to the
commit list.   Then do 
\begin{verbatim}
git commit -m "synopsis of your brilliant changes"
git push <options>
\end{verbatim}

The 'git commit' will commit your changes to your local repository.
The 'git push' can be used to contribute your changes to the github
project repository.  If you are an authorized contributor 'git push'
with no arguments will do the trick.   At this stage, I'm not clear on
what is required to become an authorized non-Microsoft contributor.

\end{document}
